    \addcontentsline{toc}{section}{Appendix}
		\large
			\begingroup
				\let\clearpage\relax
				\chapter*{Appendix}
			\endgroup
        \normalsize
			\appendix
\section*{Appendix}

\subsection*{Appendix A: Project Overview}

This appendix provides additional supporting information related to the proposed development and implementation of the \textit{StudyBug} study website. StudyBug is designed as a student-centric productivity platform aimed at improving focus, time management, and study consistency through an integrated and aesthetically pleasing interface. The appendix complements the main chapters by offering further clarification on design decisions, planned development practices, and intended system usage.

\subsection*{Appendix B: Tools and Technologies to be Used}

The StudyBug project will be developed using standard web technologies to ensure accessibility and ease of deployment. HTML, CSS, and JavaScript (React) will be used for frontend development, enabling responsive design across desktop and mobile browsers. Figma will be used during the ideation and prototyping phase to design wireframes and interactive layouts before implementation. GitHub will be used for version control and collaborative development, allowing team members to work on assigned tasks efficiently and maintain code integrity.

\subsection*{Appendix C: Planned Development Workflow}

The development process will follow a structured yet flexible approach. Initially, project requirements will be analyzed and tasks will be divided among team members. A shared GitHub repository will be created, and development will be carried out using feature-based contributions. Each team member will implement their assigned components locally and push updates to the repository. Changes will be reviewed and merged into the main branch to ensure smooth integration and consistency across the system.

\subsection*{Appendix D: Intended System Usage}

StudyBug will allow users to create and manage study tasks, plan daily or weekly study schedules, and conduct focused study sessions using a built-in timer. The system will record completed tasks and study sessions to provide basic progress tracking. Users will also be able to customize the visual appearance of the website to create a comfortable study environment. Data will be stored locally within the browser, enabling continued usage without mandatory login or backend dependency.

\subsection*{Appendix E: Limitations and Assumptions}

The proposed implementation of StudyBug will focus on core study-support features and will not include advanced analytics, cloud synchronization, or collaborative study features in the initial version. The system assumes that users will access the website through a modern web browser with JavaScript enabled. Future versions of the project may address these limitations by incorporating additional features and backend support.

