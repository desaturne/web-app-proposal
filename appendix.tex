    \addcontentsline{toc}{section}{Appendix}
		\large
			\begingroup
				\let\clearpage\relax
				\chapter*{Appendix}
			\endgroup
        \normalsize
			\appendix
\section*{Appendix}

\subsection*{Appendix A: Project Overview}

This appendix provides additional supporting information related to the development and implementation of the \textit{StudyBug} study website. StudyBug is designed as a student-centric productivity platform aimed at improving focus, time management, and study consistency through an integrated and aesthetically pleasing interface. The appendix complements the main chapters by offering further clarification on design decisions, development practices, and system usage.

\subsection*{Appendix B: Tools and Technologies Used}

The StudyBug project was developed using standard web technologies to ensure accessibility and ease of deployment. HTML, CSS, and JavaScript were used for frontend development, enabling responsive design across desktop and mobile browsers. Figma was used during the ideation and prototyping phase to design wireframes and interactive layouts before implementation. GitHub was used for version control and collaborative development, allowing team members to work on assigned tasks efficiently and maintain code integrity.

\subsection*{Appendix C: Development Workflow}

The development process followed a structured yet flexible approach. Initially, project requirements were analyzed and tasks were divided among team members. A shared GitHub repository was then created, and development was carried out using feature-based contributions. Each team member implemented their assigned components locally and pushed updates to the repository. Changes were reviewed and merged into the main branch to ensure smooth integration and consistency across the system.

\subsection*{Appendix D: System Usage Description}

StudyBug allows users to create and manage study tasks, plan daily or weekly study schedules, and conduct focused study sessions using a built-in timer. The system records completed tasks and study sessions to provide basic progress tracking. Users can also customize the visual appearance of the website to create a comfortable study environment. Data is stored locally within the browser, enabling continued usage without mandatory login or backend dependency.

\subsection*{Appendix E: Limitations and Assumptions}

The current implementation of StudyBug focuses on core study-support features and does not include advanced analytics, cloud synchronization, or collaborative study features. The system assumes that users access the website through a modern web browser with JavaScript enabled. Future versions of the project may address these limitations by incorporating additional features and backend support.

