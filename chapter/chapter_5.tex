      \chapter{Requirement Analysis}
        
        \section{Software Requirement}
            The proposed system is a web-based application and requires commonly available software for both development and usage. To access the website, users require a modern web browser such as Google Chrome, Mozilla Firefox, or Microsoft Edge. Overall, the required software tools are easily accessible and widely supported.
         \section{Hardware Requirement}
            Since the system is web-based, the hardware requirements are minimal. For development purposes, a standard laptop or desktop computer with sufficient processing capability and a stable internet connection is adequate. For end users, any device capable of running a modern web browser—such as a laptop, tablet, or smartphone—can be used to access the website. No specialized hardware is required, making the system accessible to a wide range of users.
            
        \section{Functional Requirement}
            The aesthetic study website is designed to assist users in improving focus, organization, and study consistency. The system allows users to create and manage study tasks, organize daily and weekly study schedules, and conduct focused study sessions using a built-in timer. Users can start, pause, and reset study sessions as needed. The system records completed tasks and study sessions and displays progress summaries to help users track their productivity. Additionally, the website provides an aesthetic and distraction-minimized interface to enhance motivation. Users can personalize the appearance of the website through theme or layout options, and study data is stored either locally or in a database to allow continued usage over time. Optional login functionality may be provided to enable access across multiple devices.

        \section{Non-Functional Requirement}
            Non-functional requirements define the quality attributes of the system. The website should be reliable and function smoothly without frequent errors or data loss during normal usage. The system should be easy to maintain, with modular code that allows future enhancements and feature additions. Performance is also an important consideration; the website should load quickly and respond efficiently to user interactions such as scheduling tasks or starting focus sessions. Finally, usability is a key requirement, as the interface should be intuitive, visually appealing, and responsive across different screen sizes to ensure a positive user experience for all users.