    \chapter{Feasibility Study}
        \section{Technical Feasibility}
            The proposed aesthetic study website is technically feasible as all required features can be implemented using standard web technologies. Features such as task planning, scheduling, focus timers, progress tracking, reminders, and theme customization can be developed using HTML, CSS, JavaScript, and modern frameworks like React, Vue, or Angular. The system can support multiple devices through a responsive design. Data storage can be handled using client-side storage for offline use or databases such as Firebase or MongoDB for persistent, login-based access. Secure authentication and deployment using platforms like GitHub Pages, Netlify, or Vercel further support technical feasibility.
            
        \section{Operational Feasibility}
           Operational feasibility assesses how well the system will work in real conditions and how easily users can adopt it.
            The proposed website is operationally feasible because it addresses real student challenges: lack of focus, unstructured schedules, and low motivation. The system is designed to be simple and practical—users can plan tasks, schedule study time, run focus sessions, and track progress within one platform. Since the website is accessible through any browser, users do not need installation, which improves accessibility and adoption.
            Module-wise operational feasibility is described below:
            
           \subsection{Focus Timer and Study Session Module}
                This module supports focused study techniques such as Pomodoro. Users can start and pause study sessions, take short breaks, and record study time. A distraction-minimized focus view improves usability. Since timers and session counters are basic web functions, this module can operate reliably in a browser environment.
           \subsection{Study Scheduling and Planner Module}
               The scheduling module helps users create daily or weekly study plans and allocate time blocks for subjects or tasks. This improves structure and reduces procrastination. The module is operationally practical as users can edit schedules easily, and the interface can present schedules in a clean and understandable format.
            \subsection{Aesthetic UI, Themes, and Personalization}
                This module focuses on the visual appeal of the website through themes, layouts, fonts, and calming design elements. Aesthetic design improves engagement and increases long-term use by making the study environment more pleasant. This supports the core idea of the project: studying should feel less stressful and more enjoyable.

        \section{Economic Feasibility}
            The project is economically feasible because it can be developed using free and open-source tools. Hosting options such as GitHub Pages, Netlify, and Vercel offer free or low-cost deployment, while backend services are available on free tiers if required. Development costs are minimal and mainly involve time and effort, making the project suitable for an academic budget.

        \section{Scheduling Feasibility}
            The proposed aesthetic study website is schedule-feasible as it can be developed within an academic semester using a phased and modular approach. Initial weeks can be dedicated to requirement analysis, literature review, and user interface planning, followed by wireframe design and theme selection. Core development, including task management, study scheduling, focus timers, and data storage, can be completed in the middle phase of the timeline. Subsequent weeks can focus on progress tracking features, usability testing, responsive design improvements, and bug fixing. The final phase can be used for documentation, deployment, and presentation preparation. Due to the manageable scope and clear module separation, the project can be completed within the available time without schedule overruns.


