\chapter{Literature Review}

\section{Assessing the efficacy of the Pomodoro technique in enhancing anatomy lesson retention during study sessions: a scoping review}
In this paper, \citep{Ogut2025} presented a scoping review analyzing the impact of the Pomodoro Technique (PT) on cognitive performance and retention, specifically within the context of anatomy education. The author highlighted that anatomy requires substantial cognitive effort, often leading to mental fatigue when students rely on self-paced study habits. The review found that structured Pomodoro intervals (typically 25 minutes of work followed by 5-minute breaks) were significantly more effective than unstructured, self-regulated breaks.

The study reported that students using the Pomodoro technique experienced approximately 20\% lower fatigue levels and a 15–25\% increase in self-rated focus compared to control groups. Furthermore, the use of digital tools and timers to enforce these intervals was found to enhance student engagement by 10–18\%. The authors concluded that time-structured interventions consistently outperformed self-paced study sessions by reducing distractibility and sustaining motivation over longer periods. This suggests that integrating Pomodoro timers into productivity applications can serve as a critical mechanism for preventing cognitive overload and improving long-term retention of complex material.

\section{Analyzing the Impact of AI Tools on Student Study Habits and Academic Performance}
The integration of mobile technology in higher education has created new opportunities for enhancing student productivity and self-regulation. \cite{Hinze2023} conducted a comprehensive survey involving 269 academic staff and higher-degree students, revealing that while nearly 95\% of students possess smartphones, the use of mobile applications for academic purposes is largely driven by personal motivation rather than institutional requirements. The study found that current app usage is predominantly focused on basic document storage and communication tools such as Dropbox, rather than specialized academic process-management applications. However, a significant gap exists between current usage patterns and student needs. \cite{Hinze2023} reported that students specifically recommended applications for project and assignment planning, and non-users expressed a strong willingness to adopt academic apps if they were easier to use and more appropriate for their academic context. These findings indicate a clear demand for simplified, student-centric productivity applications that extend beyond basic file storage.

To complement the structural benefits of planning applications, recent research highlights the cognitive advantages of time-management techniques such as the Pomodoro method. \cite{Ogut2025} presented a scoping review examining the impact of the Pomodoro Technique on anatomy students, a population frequently exposed to high cognitive load. The review demonstrated that structured study intervals, typically consisting of 25 minutes of focused work followed by 5-minute breaks, were significantly more effective than unstructured, self-regulated breaks. Students using this technique reported approximately 20\% lower fatigue levels and a 15--25\% increase in self-rated focus compared to control groups. Importantly, \cite{Ogut2025} emphasized that digital tools and timers were critical in enforcing these intervals, resulting in a 10--18\% increase in student engagement. Taken together, these studies suggest that a productivity application combining autonomous planning features with structured focus techniques such as Pomodoro timers would effectively address the gaps in academic app usage identified by \cite{Hinze2023}.

\section{A Study of Mobile App Use for Teaching and Research in Higher Education}
In a comprehensive study on the integration of mobile technology in universities, \cite{Hinze2023} surveyed 269 academic staff and higher-degree students to examine how mobile applications are utilized in academic settings. The study established that the hardware barrier to mobile learning is virtually non-existent, with recent data indicating that approximately 95\% of students possess smartphones. Despite this widespread availability, the authors found that mobile app usage is primarily driven by \emph{personal motivation} rather than institutional planning. This suggests that students independently select digital tools to manage their academic activities, thereby validating the demand for student-centric productivity applications.

The study further revealed that the current academic app ecosystem is dominated by basic utility tools. Applications for document storage, such as Dropbox, and communication were reported as the most frequently used, whereas specialized study or process-management tools were comparatively underutilized. However, the findings also highlighted a clear demand for applications that support structured academic work. When participants were asked to recommend purposes for academic app usage, \emph{project and assignment planning} emerged as a key category. Additionally, among participants who did not currently use mobile apps for academic purposes, a significant proportion expressed an intention to adopt such tools in the future, particularly for project or assignment planning and note-taking.