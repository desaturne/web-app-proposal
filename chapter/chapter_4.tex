\chapter{Methodology}
	\section{Agile Method as Software Development Model}
	The development of the Aesthetic Study Website follows the Agile software development methodology. Agile was chosen because the project emphasizes user experience, iterative design, and continuous improvement. Since the website includes multiple interactive features such as scheduling, focus timers, task management, and aesthetic customization, Agile allows these components to be developed and refined incrementally.
The Agile approach enables frequent evaluation of progress, early detection of issues, and flexibility in incorporating feedback. Each iteration focuses on delivering a functional and usable version of the website, which is then enhanced in subsequent iterations.
\\
	\begin{figure}[h]
		\centering
			\includegraphics[width=0.8\textwidth]{img/agilem.png}
			\caption{Agile Model as Software Development Model}    
		\end{figure}
	\break
	\section{Figma for Ideation and Prototyping}
	Figma was used for ideation and prototyping to design wireframes and interactive layouts of the StudyBug website. It helped visualize user flows, refine the interface, and ensure an aesthetic and user-friendly design before development.\\
	\begin{figure}[h]
		\centering
			\includegraphics[width=1\textwidth]{img/figmapro.png}
			\caption{Figma Workflow}    
	\end{figure}
	The Project will be managed in Notion in five sessions:
		\subsection{Project Plan}
			Planning for further steps will be done in detail in this session. All the tasks to be done, their sub tasks, deadline of the tasks and subtasks will be recorded here. Then the task will be linked to respective iteration session by the assigned member before doing the task.
		\subsection{Iteration1}
			Everything related to the first iteration of our project will be recorded here including meeting minutes and reports.
		\subsection{Iteration2}
			Everything related to the second iteration of our project will be recorded here including meeting minutes and reports.
	\section{Overall Phase to be Followed}
	The overall project will be completed in three main phases which are:
	\begin{enumerate}
		\item Planning Phase
		\item Development Phase
		\item Integration
	\end{enumerate}
	\subsection{Planning Phase}
	The planning phase involved identifying the problem statement, defining project objectives, and gathering requirements based on student study challenges. During this phase, features were finalized and the overall structure of the StudyBug website was designed using Figma wireframes.\\
	First the project was divided into four parts:
	\begin{enumerate}
		\item UI/UX design
		\item Frontend
		\item Backend
		\item Database
	\end{enumerate}
	These parts were then assigned to each project members who then studied about the respective parts in detail.
	\subsection{Development Phase}
	In this phase, the divided parts will be studied and developed. Then each of those developed parts will be tested separately.
	\subsection{Integration}
	In this phase, the separately developed parts will be integrated to form a system and integration testing will be done.\pagebreak
	\section{Task Workflow}
	Each task of every session will be done and recorded in a procedural manner and the programmes will be recorded and stored in GitHub.\\
	The workflow of the tasks will follow these steps:
	\begin{enumerate}
		\item Get task assigned on the basis of agreement.
		\item A shared GitHub repository will be created for the StudyBug project.
		\begin{figure}[h]
			\begin{center}
				\includegraphics[width=\textwidth]{img/gitlab_create_issue.png}
			\end{center}
			\caption{Create a shared github repository}    
		\end{figure}\\
		\item Each team member will work on their assigned tasks using separate branches.
        \item Changes will be implemented locally and pushed to the GitHub repository.
		\item Completed work will be reviewed and merged into the main branch for integration.
	\end{enumerate}