     \chapter{Introduction}
        \pagenumbering{arabic}
        \section{Background Introduction}
        In today’s academic environment, students are expected to manage heavy workloads, tight deadlines, and continuous assessments. Effective studying requires concentration, proper planning, and consistency. However, many students struggle to maintain focus during study sessions due to distractions such as mobile phones, social media, and mental fatigue. In addition to focus-related issues, the absence of a structured study schedule often leads to procrastination, stress, and inefficient learning. \\
            \\
            Although several digital tools such as calendars, to-do lists, and focus timers exist, most of them are either too complex or lack long-term engagement. Many study applications focus solely on functionality and ignore the emotional and visual aspects that influence motivation. As a result, students often abandon these tools after short-term use. \\
            \\
            To address these challenges, this project proposes the development of an aesthetic study app that combines focus enhancement, structured scheduling, and an engaging visual experience. The goal is to create a study environment that not only improves productivity but also makes studying more enjoyable and sustainable.
        \section{Motivation}
        The motivation for this project originates from our personal experience as students. While studying, we frequently faced loss of focus during study sessions, lack of a proper and consistent study schedule and reduced motivation because studying felt monotonous and stressful
            \\
            We observed that even when study resources were available, the absence of an engaging and organized system made it difficult to stay disciplined. Traditional study methods and existing apps felt repetitive and uninteresting, which further reduced consistency.
            \\
            This realization led to the idea of developing a study app that is visually aesthetic, structured, and enjoyable to use. By making the study process more appealing and organized, the app aims to help students stay focused, manage time better, and develop healthier study habits.
        \section{Problem Definition}
        Despite the availability of numerous productivity and study-related applications, many students continue to struggle with maintaining focus and consistency in their academic routines.     \\
            Common challenges include frequent distractions during study sessions, ineffective time management due to a lack of structured scheduling, and low engagement with existing tools that fail to provide visual or emotional motivation. These issues highlight the need for an integrated study application that not only supports focused study and effective planning but also enhances motivation through an aesthetic, intuitive, and user-friendly interface.
        \section{Goals and Objectives}
            The main objective of this project is:
            \begin{itemize}
                \item To analyze common problems faced by students during studying.
            \end{itemize}
        \section{Scope and Applications}
        The major scope of the project is to assist visually impaired people and the system is able to:
        \begin{itemize}
            \item study scheduling and task planning
            \item aesthetic UI design to improve motivation and user engagement
        \end{itemize}
        \section{Report Organization}
            \subsection{Introduction}
                The main purpose of the introduction is to set the scene for our readers so that they can know about the problems that visually impaired people have to go through and how our purposed system can help them. 
            \subsection{Literature Review}
                The literature review is there to familiarize ourselves with the current state of knowledge on assistance for visually impaired people and ensure us to not repeat what others have already done. 
            \subsection{Feasibility Study}
                It is used to determine the viability of an idea, such as ensuring our project is legally and technically feasible as well as economically justifiable or not. 
            \subsection{Methodology}
                It critically helps us to analyze and select correct method for our project to avoid unnecessary hurdles. 
            \subsection{Requirement Analysis}
                It gives us a clear vision about the necessary programming languages and software required to build voice assistance. 
            \subsection{System Design and Architecture}
                The main purpose of this is to evaluate the contribution of each component for overall performance of the system using different diagrams. 
            \subsection{Block Diagram and Description of Proposed System}
                It provides us quick and high-level view of different topics which enhances our ability to understand that topic.
            \subsection{Expected Outcome}
                It helps to ensure that our set goal is achieved and not to repeat what others have done already in this topic. 
            \subsection{Actual Outcome}
                It provides clear view about the accomplishments we have achieved and comparison can be made with expected outcome to note down deviations.
            \subsection{Conclusion and Future enhancements}
                The conclusion section summarizes our main thoughts on project and future enhancements provides us scope to upgrade our project. 
            